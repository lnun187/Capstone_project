\chapter{Kết quả thực nghiệm}

Trong chương này, đề tài trình bày các kết quả đạt được sau quá trình thiết kế, tổng hợp và triển khai hệ thống SoC trên nền tảng phần cứng thực tế. Nội dung bao gồm đánh giá tài nguyên sử dụng, kết quả kiểm tra các khối ngoại vi, hiệu năng của Video Streaming và quy trình khởi động hệ thống thông qua Bootloader.

\section{Môi trường thực nghiệm phần cứng}

Toàn bộ hệ thống SoC được triển khai và đánh giá trên kit phát triển \textbf{Xilinx Virtex-7 VC707}. Đây là nền tảng FPGA hiệu năng cao, cung cấp đầy đủ các giao diện cần thiết cho việc thực nghiệm.

Các thông số thiết lập xung nhịp hệ thống bao gồm:
\begin{itemize}
    \item Xung nhịp hệ thống chính (\texttt{sys\_clk}): 200 MHz.
    \item Xung nhịp cho bộ điều khiển bộ nhớ và logic xử lý: 150 MHz.
    \item Xung nhịp hiển thị (\texttt{pixel\_clk}): 25 MHz và 50 MHz (phục vụ các độ phân giải khác nhau).
\end{itemize}

\section{Kết quả hiện thực các khối giao tiếp ngoại vi}

Hệ thống đã hoàn thiện việc tích hợp và kiểm tra độ tin cậy của các chuẩn giao tiếp nối tiếp quan trọng, đảm bảo khả năng tương tác toàn diện với các thiết bị ngoại vi.

\subsection{Giao tiếp UART}
Khối UART đã được thực nghiệm thành công với tốc độ baud chuẩn 115200 bps. Kết quả kiểm tra thông qua phần mềm Terminal (PuTTY) cho thấy dữ liệu được truyền nhận chính xác giữa SoC và máy tính cá nhân, phục vụ tốt cho việc xuất log gỡ lỗi và tương tác lệnh.

\subsection{Giao tiếp I2C và SPI}
\begin{itemize}
    \item \textbf{I2C:} Được sử dụng để cấu hình các thanh ghi nội của cảm biến hình ảnh OV5640 và chip phát HDMI ADV7513. Thực nghiệm cho thấy quá trình cấu hình diễn ra ổn định, các thiết bị phản hồi đúng địa chỉ và mã lệnh.
    \item \textbf{SPI:} Đã hiện thực hóa bộ điều khiển SPI Master để giao tiếp với các cảm biến phụ trợ và bộ nhớ Flash trên board.
\end{itemize}

\subsection{Giao tiếp Octal-SPI (OSPI) hỗ trợ DDR}
Khối OSPI đã được tối ưu hóa để hỗ trợ chế độ \textbf{Double Data Rate (DDR)}, cho phép truyền tải dữ liệu trên cả hai cạnh của xung nhịp. Thực nghiệm xác nhận băng thông truyền tải tăng gấp đôi so với chế độ SDR truyền thống, đáp ứng yêu cầu truy xuất dữ liệu nhanh từ bộ nhớ ngoài.

\section{Kết quả phân hệ Video Streaming 60Hz}

Một trong những kết quả trọng tâm của đề tài là việc hiện thực hóa luồng dữ liệu Video thời gian thực từ camera ra màn hình.

\begin{itemize}
    \item \textbf{Tốc độ khung hình:} Hệ thống đạt mức hiển thị ổn định \textbf{60 Hz}.
    \item \textbf{Độ trễ:} Nhờ vào cơ chế quản lý bộ đệm khung hình (Frame Buffer) bằng BRAM và bộ điều khiển DMA tự thiết kế, hiện tượng xé hình (\textit{tearing}) và trễ tích lũy đã được triệt tiêu đáng kể.
    \item \textbf{Chất lượng hình ảnh:} Tín hiệu xuất ra qua cổng HDMI rõ nét, không có điểm ảnh lỗi, minh chứng cho sự đồng bộ chính xác giữa các miền xung nhịp PCLK của camera và Pixel Clock của HDMI.
\end{itemize}

\section{Hiện thực chương trình Bootloader qua SPI Flash}

Để tăng tính độc lập cho SoC, chương trình \textbf{Bootloader} đã được thiết kế để nạp từ bộ nhớ SPI Flash ngoại vi.
\begin{itemize}
    \item \textbf{Quy trình:} Khi hệ thống khởi động (Power-on Reset), lõi PicoRV32 sẽ thực thi mã lệnh từ vùng nhớ ROM khởi tạo, thực hiện đọc dữ liệu thực thi từ SPI Flash và nạp vào bộ nhớ lệnh (IMEM).
    \item \textbf{Kết quả:} Hệ thống có khả năng tự khởi động và chạy các ứng dụng phần mềm mà không cần kết nối trực tiếp với máy tính thông qua JTAG sau mỗi lần reset.
\end{itemize}

\section{Đánh giá tài nguyên sử dụng trên FPGA}

Dựa trên báo cáo tổng hợp từ công cụ Vivado cho thiết kế trên board VC707, tài nguyên hệ thống được tối ưu hóa ở mức rất thấp, cho phép mở rộng thêm các bộ gia tốc AI phức tạp trong tương lai:

\begin{table}[H]
    \centering
    \caption{Bảng tổng hợp tài nguyên sử dụng trên VC707}
    \begin{tabular}{|l|c|}
        \hline
        \textbf{Loại tài nguyên} & \textbf{Tỉ lệ sử dụng (\%)} \\ \hline
        Logic (LUT) & 1\% \\ \hline
        Thanh ghi (Flip-Flop) & 1\% \\ \hline
        Bộ nhớ khối (BRAM) & 15\% \\ \hline
        Bộ tính toán (DSP) & 1\% \\ \hline
    \end{tabular}
\end{table}

Việc chỉ sử dụng 15\% BRAM trong khi vẫn duy trì được luồng Video 60Hz và đầy đủ các ngoại vi cho thấy hiệu quả cao của kiến trúc quản lý bộ đệm và hệ thống Bus AXI4-Lite đã thiết kế.