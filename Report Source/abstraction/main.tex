% \section*{\Huge Abstract}

% Trajectory optimization offers mature tools for motion planning in high-dimensional spaces under dynamic constraints. However, in obstacle-cluttered environments, the non-convexity of the free-space typically forces roboticists to rely on sampling-based planners, which struggle with high dimensions and differential constraints. Here, we demonstrate that convex optimization can reliably solve these problems through the convex decomposition of the free space.
% Specifically, we decompose the collision-free configuration space into finite convex regions, organizing them into a Graph of Convex Sets (GCS). By combining Bézier curves with this graph structure, we formulate the motion planning problem as a compact mixed-integer convex program that naturally handles collision avoidance, velocity, and duration constraints.
% We validate GCS across a spectrum of environments with increasing complexity, ranging from simple 2D obstacles and cluttered mazes to dynamical quadrotors and high-dimensional 7-DoF manipulators. Numerical experiments demonstrate that GCS consistently outperforms widely-used sampling-based planners (e.g., PRM). 
% Specifically, in high-dimensional tasks, GCS reduces online query time by up to an order of magnitude compared to standard PRM, while finding globally optimal trajectories that are significantly shorter (up to 40\% reduction in length) than those from post-processed sampling-based planners~\cite{motionplanning2022}. Ultimately, this work brings the reliability and speed of convex optimization to the traditionally challenging domain of nonconvex motion planning.


\section*{\Huge Tóm tắt}

Hoạch định chuyển động cho robot trong môi trường chứa vật cản vốn là một bài toán tối ưu hóa phi tuyến đầy thách thức do tính chất không lồi của không gian cấu hình. Trong khi các phương pháp dựa trên lấy mẫu truyền thống thường gặp hạn chế về tính tối ưu và khả năng xử lý các ràng buộc động lực học phức tạp, bài báo này đề xuất một phương pháp tiếp cận mới dựa trên Đồ thị các Tập Lồi (Graphs of Convex Sets - GCS). Bằng cách phân rã không gian tự do thành các vùng lồi hữu hạn và mô hình hóa quỹ đạo robot sử dụng đường cong Bézier, chúng tôi thiết lập bài toán dưới dạng tối ưu hóa hỗn hợp nguyên. Cuối cùng ta áp dụng kỹ thuật nới lỏng lồi (convex relaxation) chặt chẽ để chuyển đổi bài toán phức tạp này thành bài toán Lập trình nón bậc hai (Second-Order Cone Programming - SOCP) dễ giải. Kết quả thực nghiệm trên các hệ thống có số chiều cao chứng minh phương pháp này vượt trội hơn so với các thuật toán PRM truyền thống, giảm thiểu đáng kể thời gian tính toán trong khi vẫn đảm bảo tìm được quỹ đạo tối ưu toàn cục. 
Chúng tôi kiểm chứng GCS trên một loạt các môi trường với độ phức tạp tăng dần, từ các vật cản 2D đơn giản và mê cung phức tạp đến các thiết bị bay quadrotor động lực học và tay máy 7 bậc tự do (7-DoF) trong không gian nhiều chiều. Các thực nghiệm số chứng minh rằng GCS đạt hiệu năng vượt trội một cách nhất quán so với các bộ hoạch định dựa trên lấy mẫu được sử dụng rộng rãi (ví dụ: PRM). Cụ thể, trong các nhiệm vụ nhiều chiều, GCS giảm thời gian truy vấn trực tuyến lên đến một bậc so với PRM tiêu chuẩn, đồng thời tìm ra các quỹ đạo tối ưu toàn cục ngắn hơn đáng kể (giảm tới 40\% độ dài) so với các bộ hoạch định dựa trên lấy mẫu đã được xử lý hậu kỳ~\cite{motionplanning2022}. 