\chapter{Kết luận và Hướng phát triển}
\label{ch:future}

\textit{Chương này tổng kết các kết quả đạt được và chưa đạt được trong Giai đoạn 1 và đề ra kế hoạch chi tiết cho việc hiện thực và kiểm thử trong Giai đoạn 2.}

% \section{Đánh giá mức độ hoàn thành Giai đoạn 1}
% \input{7_ke_hoach_phat_trien/7.1_danh_gia_giai_doan_1.tex}

% \section{Kế hoạch thực hiện Giai đoạn 2}
% \input{7_ke_hoach_phat_trien/7.2_ke_hoach_giai_doan_2.tex}

% \section{Tiến độ dự kiến}
% \input{7_ke_hoach_phat_trien/7.3_tien_do_du_kien.tex}


% % File: 7_ket_luan_huong_phat_trien.tex

% \chapter{Kết luận và Hướng phát triển}
% \label{ch:future}

% \textit{Chương này tổng kết các kết quả đạt được trong Giai đoạn 1 của đề tài, đồng thời phân tích những hạn chế hiện hữu để làm cơ sở đề ra kế hoạch thực hiện và kiểm thử chi tiết cho Giai đoạn 2.}

\section{Đánh giá Giai đoạn 1}

Qua giai đoạn 1, đề tài đã cơ bản hoàn thiện phần cứng cốt lõi cho hệ thống SoC RISC-V. Thành phần trung tâm của hệ thống là lõi vi xử lý PicoRV32 đã vận hành ổn định trên nền tảng FPGA Virtex-7 VC707 tại tần số 200 MHz, được kết nối thông qua hệ thống Bus AXI4-Lite Interconnect linh hoạt hỗ trợ cấu hình đa Master - đa Slave. Bên cạnh đó, các khối ngoại vi giao tiếp thiết yếu đã được tích hợp và kiểm thử thành công, bao gồm UART, I2C Master phục vụ cấu hình cảm biến, và SPI Master. Đặc biệt, việc hiện thực hóa khối Octal-SPI (OSPI) hỗ trợ chế độ Double Data Rate (DDR) giao tiếp trực tiếp với chip nhớ W958D8NBYA ở tần số 50 MHz đã khẳng định khả năng đáp ứng băng thông lớn cho các tác vụ xử lý dữ liệu.
\\

Về khả năng vận hành độc lập, đề tài đã xây dựng hoàn tất chương trình Bootloader nạp từ SPI Flash, cho phép hệ thống tự khởi động mã lệnh thực thi mà không cần sự tổng hợp lại trên FPGA. Đối với hệ thống Video Streaming, hệ thống đã bước đầu thu nhận được dữ liệu từ cảm biến camera OV5640 và hiển thị lên màn hình qua cổng HDMI với tốc độ khung hình 60 Hz. Tuy nhiên, qua thực nghiệm cho thấy luồng dữ liệu vẫn còn tồn tại hiện tượng lỗi đồng bộ (\textit{synchronization issues}), gây ảnh hưởng đến độ ổn định của hình ảnh trong một số điều kiện vận hành. Song song đó, khối gia tốc AI đã được định hình kiến trúc và thiết kế ổn định về mặt cấu trúc logic, nhưng vẫn cần thực hiện các tinh chỉnh bổ sung để tối ưu hóa hiệu suất tính toán và làm rõ ràng hơn các luồng dữ liệu nội bộ. Với mức độ sử dụng tài nguyên FPGA hiện tại chỉ đạt 15\% BRAM và 1\% LUT/FF, hệ thống vẫn còn dư địa rất lớn để tích hợp các tính năng phức tạp hơn trong tương lai.

\section{Kế hoạch thực hiện Giai đoạn 2}

Dựa trên những kết quả đã đạt được và các hạn chế còn tồn tại, kế hoạch cho Giai đoạn 2 tập trung vào việc hoàn thiện hiệu năng và tích hợp trí tuệ nhân tạo vào SoC. Trọng tâm hàng đầu là hiện thực hóa khối điều khiển truy cập bộ nhớ trực tiếp (DMA) chuyên dụng để quản lý việc truyền tải dữ liệu tốc độ cao giữa bộ đệm khung hình và khối gia tốc, qua đó giảm thiểu tải xử lý cho CPU và tối ưu hóa băng thông Bus. Đồng thời, hệ thống Video Streaming sẽ được tinh chỉnh kỹ lưỡng về mặt đồng bộ hóa miền xung nhịp (\textit{Clock Domain Crossing}) để khắc phục triệt để các lỗi hiển thị hiện tại, đảm bảo dòng video mượt mà phục vụ cho các thuật toán nhận diện.

Sau khi đường truyền dẫn dữ liệu hình ảnh ổn định, đề tài sẽ tiến hành hoàn thiện và nhúng khối gia tốc AI vào hệ thống SoC. Tập trung vào việc tối ưu hóa tài nguyên tính toán. Mục tiêu cuối cùng là thực hiện quá trình tối ưu hóa thiết kế hướng tới ASIC, đồng thời cho ra đời một sản phẩm SoC tích hợp AI hoàn chỉnh và có khả năng demo thực tế các ứng dụng Edge AI trên nền tảng FPGA một cách thuyết phục.

\section{Tiến độ dự kiến}

Lộ trình thực hiện Giai đoạn 2 dự kiến kéo dài trong 12 tuần với các cột mốc quan trọng được quy hoạch như sau:

\begin{table}[H]
    \centering
    \caption{Bảng tiến độ thực hiện Giai đoạn 2}
    \renewcommand{\arraystretch}{1.3}
    \begin{tabular}{|c|p{9cm}|c|}
        \hline
        \textbf{Tuần} & \multicolumn{1}{|c|}{\textbf{Nội dung công việc}} & \textbf{Ghi chú} \\ \hline
        1 - 3 & Thiết kế, kiểm thử khối DMA và khắc phục lỗi đồng bộ Video Stream. & Trọng tâm \\ \hline
        1 - 6 & Tinh chỉnh cấu trúc, tối ưu hóa và kiểm thử độc lập khối gia tốc AI. & Kỹ thuật \\ \hline
        7 - 8 & Tích hợp toàn hệ thống (SoC + AI Accel + DMA + Video). & Trọng tâm \\ \hline
        9 - 10 & Phát triển hệ sinh thái phần mềm, driver điều khiển và ứng dụng mẫu. & Phần mềm \\ \hline
        11 - 12 & Tối ưu hóa định thời hệ thống, đóng gói sản phẩm và chuẩn bị báo cáo. & Hoàn tất \\ \hline
    \end{tabular}
\end{table}

Việc bám sát lộ trình này sẽ đảm bảo hệ thống đạt được sự cân bằng tối ưu giữa hiệu năng phần cứng và tính linh hoạt của phần mềm, hướng tới một giải pháp SoC RISC-V tích hợp Edge AI mạnh mẽ và khả thi trong các ứng dụng IoT thực tiễn.