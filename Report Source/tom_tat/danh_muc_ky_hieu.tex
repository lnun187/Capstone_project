\chapter*{Danh mục Ký hiệu và Chữ viết tắt}
\addcontentsline{toc}{chapter}{Danh mục Ký hiệu và Chữ viết tắt}
\label{ch:symbols}

\begin{longtable}{p{2.5cm} p{12.5cm}}
    \textbf{Ký hiệu} & \textbf{Ý nghĩa} \\
    \hline
    \endhead
    
    $N$ & Kích thước lô (Batch size) \\
    $C$ & Số lượng kênh đầu vào (Input Channels) \\
    $M$ & Số lượng kênh đầu ra (Output Channels/Filters) \\
    $H_{in}, W_{in}$ & Chiều cao và chiều rộng của đặc trưng đầu vào (Input Feature Map) \\
    $H_{out}, W_{out}$ & Chiều cao và chiều rộng của đặc trưng đầu ra (Output Feature Map) \\
    $R, S$ & Chiều cao và chiều rộng của bộ lọc (Kernel Height, Kernel Width) \\
    $U$ & Bước trượt (Stride) \\
    $P$ & Kích thước vùng đệm (Padding) \\
    $I$ & Tensor dữ liệu đầu vào \\
    $O$ & Tensor dữ liệu đầu ra \\
    $W$ & Tensor trọng số (Weights) \\
    $B$ & Vector hệ số chệch (Bias) \\
    $O_{dw}$ & Đầu ra của lớp Depthwise Convolution \\
    $O_{pw}$ & Đầu ra của lớp Pointwise Convolution \\
    $\mu$ & Giá trị trung bình (Mean) trong Batch Normalization \\
    $\sigma$ & Phương sai (Variance) trong Batch Normalization \\
    $\gamma$ & Tham số tỉ lệ (Scale factor) \\
    $\beta$ & Tham số dịch chuyển (Shift factor) \\
    $\epsilon$ & Hằng số Epsilon \\
    & \\
    \textbf{Viết tắt} & \textbf{Ý nghĩa} \\
    \hline
    AI & Trí tuệ nhân tạo (Artificial Intelligence) \\
    SoC & Hệ thống trên chip (System-on-Chip) \\
    FPGA & Mảng cổng lập trình được dạng trường (Field-Programmable Gate Array) \\
    CNN & Mạng nơ-ron tích chập (Convolutional Neural Network) \\
    DNN & Mạng nơ-ron sâu (Deep Neural Network) \\
    RTL & Mức chuyển giao thanh ghi (Register Transfer Level) \\
    IP & Sở hữu trí tuệ (Intellectual Property - Khối thiết kế phần cứng) \\
    PE & Phần tử xử lý (Processing Element) \\
    MAC & Phép tính Nhân-Cộng tích lũy (Multiply-Accumulate) \\
    DMA & Truy cập bộ nhớ trực tiếp (Direct Memory Access) \\
    AXI-Lite & Giao diện mở rộng nâng cao rút gọn (Advanced eXtensible Interface Lite) \\
    AXI-Stream & Giao diện luồng dữ liệu mở rộng nâng cao (Advanced eXtensible Interface Stream) \\
    BRAM & Block RAM (Bộ nhớ nội trên FPGA) \\
    DSP & Digital Signal Processing (Khối xử lý tín hiệu số trên FPGA) \\
    LUT & Bảng tra (Look-Up Table) \\
    FF & Flip-Flop \\
    OSPI & Giao diện ngoại vi nối tiếp 8 kênh (Octal Serial Peripheral Interface) \\
    SPI & Giao diện ngoại vi nối tiếp (Serial Peripheral Interface) \\
    UART & Bộ truyền nhận dữ liệu nối tiếp bất đồng bộ (Universal Asynchronous Receiver-Transmitter) \\
    I2C & Giao thức giao tiếp giữa các vi mạch (Inter-Integrated Circuit) \\
    DVP & Cổng dữ liệu hình ảnh kỹ thuật số (Digital Video Port) \\
    GPIO & Cổng vào/ra đa dụng (General Purpose Input/Output) \\
\end{longtable}