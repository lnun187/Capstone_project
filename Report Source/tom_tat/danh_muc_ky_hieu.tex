\chapter*{Danh mục Ký hiệu và \\ Chữ viết tắt}
\addcontentsline{toc}{chapter}{Danh mục Ký hiệu và Chữ viết tắt}
\label{ch:symbols}

\begin{longtable}{p{2.5cm} p{12.5cm}}
    \textbf{Ký hiệu} & \textbf{Ý nghĩa} \\
    \hline
    \endhead
    
    % --- Kích thước mạng nơ-ron cơ bản ---
    $H, W$ & Chiều cao và chiều rộng của đặc trưng đầu vào (Input Feature Map) \\
    $C$ & Số lượng kênh đầu vào (Input Channels) \\
    $N_f$ & Số lượng bộ lọc / Số kênh đầu ra (Number of Filters / Output Channels) \\
    $H_{out}, W_{out}$ & Chiều cao và chiều rộng của đặc trưng đầu ra (Output Feature Map) \\
    $R, S$ & Chiều cao và chiều rộng của bộ lọc (Kernel Height, Kernel Width) \\
    $P$ & Kích thước vùng đệm (Padding) \\
    $Str$ (hoặc $U$) & Bước trượt (Stride) \\
    
    % --- Tham số Phân mảnh (Tiling) - Mới thêm vào ---
    $T_h$ & Chiều cao của mảnh dữ liệu đầu vào trong một Pass (Tile Height) \\
    $T_c$ & Số kênh đầu vào được xử lý song song trong một Pass (Tile Input Channels) \\
    $T_m$ & Số bộ lọc được tính toán song song trong một Pass (Tile Output Channels) \\
    $T_{ho}$ & Chiều cao hợp lệ của mảnh dữ liệu đầu ra trong một Pass \\
    
    % --- Tham số Hiệu năng & Thời gian - Mới thêm vào ---
    $b$ & Số chu kỳ đồng hồ để truyền một giá trị dữ liệu (Cycles per Data Transfer) \\
    $T_{comp}$ & Thời gian tính toán (Computation time) \\
    $T_{load}$ & Thời gian nạp dữ liệu (Load time) \\
    $T_{store}$ & Thời gian ghi dữ liệu (Store time) \\
    $T_{pass}$ & Thời gian hoàn thành một Pass \\
    
    % --- Các ký hiệu toán học khác ---
    $I$ & Tensor dữ liệu đầu vào \\
    $O$ & Tensor dữ liệu đầu ra \\
    $W$ & Tensor trọng số (Weights) \\
    $B$ & Vector hệ số chệch (Bias) \\
    $\mu, \sigma$ & Giá trị trung bình (Mean) và Phương sai (Variance) trong Batch Norm \\
    $\gamma, \beta$ & Tham số tỉ lệ (Scale) và dịch chuyển (Shift) trong Batch Norm \\
    
    & \\
    \textbf{Viết tắt} & \textbf{Ý nghĩa} \\
    \hline
    % --- Viết tắt chung ---
    AI & Trí tuệ nhân tạo (Artificial Intelligence) \\
    CNN & Mạng nơ-ron tích chập (Convolutional Neural Network) \\
    DNN & Mạng nơ-ron sâu (Deep Neural Network) \\
    FPGA & Mảng cổng lập trình được dạng trường (Field-Programmable Gate Array) \\
    SoC & Hệ thống trên chip (System-on-Chip) \\
    RTL & Mức chuyển giao thanh ghi (Register Transfer Level) \\
    
    % --- Viết tắt chuyên ngành phần cứng (Architecture) ---
    IFM & Đặc trưng đầu vào (Input Feature Map) \\
    OFM & Đặc trưng đầu ra (Output Feature Map) \\
    PE & Phần tử xử lý (Processing Element) \\
    PU & Đơn vị xử lý (Processing Unit - Chứa nhiều PE) \\
    PA & Mảng xử lý (Process Array - Chứa nhiều PU) \\
    MAC & Phép tính Nhân-Cộng tích lũy (Multiply-Accumulate) \\
    DMA & Truy cập bộ nhớ trực tiếp (Direct Memory Access) \\
    AXI-Lite & Giao diện mở rộng nâng cao rút gọn (Advanced eXtensible Interface Lite) \\
    AXI-Stream & Giao diện luồng dữ liệu mở rộng nâng cao (Advanced eXtensible Interface Stream) \\
    BRAM & Block RAM (Bộ nhớ nội trên FPGA) \\
    DMA & Truy cập bộ nhớ trực tiếp (Direct Memory Access) \\
    AXI & Giao diện mở rộng nâng cao (Advanced eXtensible Interface) \\
    DSP & Digital Signal Processing (Khối xử lý tín hiệu số trên FPGA) \\
    LUT & Bảng tra (Look-Up Table) \\
    FF & Flip-Flop \\
    OSPI & Giao diện ngoại vi nối tiếp 8 kênh (Octal Serial Peripheral Interface) \\
    SPI & Giao diện ngoại vi nối tiếp (Serial Peripheral Interface) \\
    UART & Bộ truyền nhận dữ liệu nối tiếp bất đồng bộ (Universal Asynchronous Receiver-Transmitter) \\
    I2C & Giao thức giao tiếp giữa các vi mạch (Inter-Integrated Circuit) \\
    DVP & Cổng dữ liệu hình ảnh kỹ thuật số (Digital Video Port) \\
    GPIO & Cổng vào/ra đa dụng (General Purpose Input/Output) \\
\end{longtable}