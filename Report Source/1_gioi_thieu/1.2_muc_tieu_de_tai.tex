\section{Mục tiêu và Nhiệm vụ nghiên cứu}
\label{sec:muc_tieu}

Mục tiêu chính của đề tài là nghiên cứu, thiết kế và hiện thực một hệ thống trên chip (SoC) hoàn chỉnh tích hợp lõi vi xử lý RISC-V và bộ tăng tốc phần cứng (Hardware Accelerator) cho các tác vụ trí tuệ nhân tạo tại biên (EdgeAI). Cụ thể, đề tài hướng tới các mục tiêu sau:

\begin{itemize}
    \item \textbf{Về kiến trúc hệ thống:} Xây dựng kiến trúc SoC tối ưu năng lượng, sử dụng chuẩn giao tiếp AXI để kết nối giữa vi xử lý trung tâm và khối tăng tốc tính toán.
    \item \textbf{Về xử lý AI:} Thiết kế khối Accelerator chuyên dụng hỗ trợ các phép toán trọng yếu của mạng nơ-ron tích chập (CNN) như AlexNet, VGG16, MobileNetv1, nhằm giảm tải cho CPU và tăng tốc độ xử lý thực tế.
    \item \textbf{Về ứng dụng thực tế:} Tích hợp đầy đủ các giao tiếp ngoại vi (Camera DVP, UART, SPI) để xây dựng một ứng dụng IoT trọn vẹn (ví dụ: nhận diện vật thể hoặc phân loại ảnh) chạy trực tiếp trên nền tảng FPGA.
    \item \textbf{Về quy trình thiết kế:} Làm chủ quy trình thiết kế từ mức RTL (Verilog) đến mô phỏng (Simulation), tổng hợp (Synthesis) và kiểm tra trên phần cứng thực (FPGA Prototyping).
\end{itemize}
