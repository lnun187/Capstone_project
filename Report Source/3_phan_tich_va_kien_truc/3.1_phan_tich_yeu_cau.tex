% File: 4_phan_tich_va_kien_truc/4.1_phan_tich_yeu_cau.tex
\subsection{Yêu cầu chức năng}
Để đảm bảo mục tiêu xây dựng một hệ thống SoC hoàn chỉnh có khả năng xử lý trí tuệ nhân tạo tại biên, thiết kế cần đáp ứng bốn nhóm yêu cầu chức năng cốt lõi liên quan đến thu thập dữ liệu, tính toán chuyên dụng, giao tiếp hệ thống và hiệu năng vận hành.

Thứ nhất, đối với phân hệ xử lý hình ảnh, hệ thống được yêu cầu phải có khả năng thu thập dữ liệu video liên tục từ Camera thông qua giao diện song song \textbf{DVP} (Digital Video Port). Luồng dữ liệu này cần được đồng bộ hóa và chuyển đổi định dạng màu sắc để hiển thị trực tiếp lên màn hình qua chuẩn \textbf{HDMI} với độ phân giải tối thiểu là VGA (640x480) hoặc HD (1280x720). Yêu cầu quan trọng đặt ra là quá trình hiển thị phải diễn ra song song với quá trình xử lý, đảm bảo người dùng có thể quan sát hình ảnh thời gian thực với tốc độ khung hình ổn định từ 30 đến 60 fps.

Thứ hai, về năng lực tính toán, hệ thống phải tích hợp một bộ gia tốc phần cứng \textbf{CNN Accelerator} đóng vai trò là một thiết bị ngoại vi chuyên dụng (Memory-mapped Peripheral). Khối này chịu trách nhiệm thực thi các phép toán nhân chập (Convolution) và các hàm kích hoạt phi tuyến của mạng nơ-ron sâu. Accelerator cần có cơ chế truy cập trực tiếp vào bộ nhớ chứa dữ liệu ảnh đầu vào mà không làm gián đoạn luồng video đang hiển thị, đồng thời trả về kết quả phân lớp để vi xử lý tổng hợp.

Thứ ba, để đảm bảo tính tương thích và khả năng mở rộng như một vi điều khiển thương mại, SoC cần hỗ trợ đầy đủ các giao thức giao tiếp tiêu chuẩn công nghiệp. Cụ thể, giao thức \textbf{UART} được sử dụng cho giao diện dòng lệnh (CLI) và gỡ lỗi hệ thống; giao thức \textbf{I2C} đóng vai trò kênh điều khiển cấu hình cho các chip ngoại vi như Camera và HDMI PHY; và giao thức \textbf{SPI/OSPI} được tích hợp để giao tiếp với bộ nhớ Flash hoặc bộ nhớ RAM mở rộng (tốc độ từ 25MHz đến 100MHz), phục vụ cho việc lưu trữ trọng số mạng và chương trình cơ sở (Firmware).

Thứ tư, về chiến lược quản lý xung nhịp và hiệu năng, hệ thống được yêu cầu thiết kế theo kiến trúc đa miền tần số (Multi-Clock Domains) nhằm tối ưu hóa tài nguyên cho từng phân hệ cụ thể. Miền xung nhịp trung tâm (System Clock) điều khiển vi xử lý RISC-V và bộ gia tốc CNN được đặt mục tiêu hoạt động ở tần số \textbf{200 MHz}, đảm bảo thông lượng tính toán cao nhất cho các tác vụ AI. Đối với phân hệ Video Streaming, kiến trúc xung nhịp được phân chia thành ba tầng xử lý riêng biệt: mức \textbf{150 MHz} dành cho các khối xử lý dữ liệu video băng thông rộng và giao tiếp bộ nhớ; mức \textbf{50 MHz} và \textbf{25 MHz} phục vụ cho các giao diện hiển thị và đồng bộ hóa tín hiệu Pixel Clock theo chuẩn VESA. Việc giao tiếp giữa miền 200 MHz của SoC và các miền tần số video thấp hơn phải được thực hiện thông qua các bộ đệm FIFO bất đồng bộ và cơ chế đồng bộ hóa Clock Domain Crossing(CDC) để triệt tiêu hiện tượng Metastability.  

\subsection{Yêu cầu phi chức năng}
Bên cạnh các chức năng vận hành cơ bản, hệ thống phải tuân thủ các ràng buộc kỹ thuật nghiêm ngặt về hiệu năng thời gian thực, tần số hoạt động và quản lý tài nguyên trên nền tảng FPGA đích.

Thứ nhất, về hiệu năng xử lý, hệ thống phải đảm bảo tốc độ khung hình hiển thị ổn định ở mức \textbf{60 FPS} (khung hình/giây) tại độ phân giải mục tiêu. Độ trễ suy luận (Inference Latency) của mô hình AI phải được tối thiểu hóa để kết quả nhận dạng (như nhãn, khung bao) xuất hiện đồng bộ với vật thể đang chuyển động trên màn hình, triệt tiêu hiện tượng trễ pha (Lag) giữa hình ảnh thực tế và kết quả xử lý.

Thứ hai, về tần số hoạt động, thiết kế phải thỏa mãn các chỉ tiêu khắt khe của kiến trúc đa miền xung nhịp. Cụ thể, sau quá trình tổng hợp và hiện thực (Implementation), miền xung nhịp trung tâm (System Clock) cho vi xử lý và bộ gia tốc phải đạt tần số hoạt động ổn định \textbf{200 MHz} để tối đa hóa thông lượng tính toán. Các miền xung nhịp phụ trợ cho video (150 MHz, 50 MHz, 25 MHz) phải đảm bảo sự chính xác về định thời (Timing constraints) để duy trì sự ổn định của tín hiệu hiển thị và giao tiếp bộ nhớ.

Thứ ba, về mặt tài nguyên, thiết kế được tối ưu hóa cho nền tảng bo mạch \textbf{Xilinx VC707} (sử dụng chip Virtex-7 XC7VX485T). Mặc dù đây là dòng FPGA hiệu năng cao với tài nguyên logic dồi dào, thách thức lớn nhất nằm ở việc quản lý hiệu quả băng thông bộ nhớ. Hệ thống phải điều phối chặt chẽ quyền truy cập giữa ba tác nhân tiêu thụ băng thông lớn: Vi xử lý (nạp lệnh/dữ liệu), Bộ gia tốc (đọc/ghi ma trận đặc trưng) và Bộ điều khiển hiển thị (quét bộ đệm khung hình). Việc tối ưu hóa này nhằm đảm bảo luồng video 60 FPS luôn mượt mà ngay cả khi bộ gia tốc hoạt động ở mức tải cao nhất.