\chapter{Phân tích và \\ Kiến trúc hệ thống}
\label{ch:architecture}

\textit{Dựa trên cơ sở lý thuyết đã trình bày, chương này đi sâu vào phân tích các yêu cầu kỹ thuật, từ đó đề xuất kiến trúc tổng thể của hệ thống SoC (System-on-Chip). Đồng thời, chương này cũng xác định đặc tả chức năng của từng khối thành phần và quy hoạch không gian địa chỉ bộ nhớ (Memory Map) cho toàn hệ thống.}

\section{Phân tích yêu cầu thiết kế}
% Phần này liệt kê:
% 1. Yêu cầu chức năng: Hệ thống phải làm gì? (Thu thập ảnh từ DVP, Xử lý qua CNN, Xuất kết quả UART/HDMI...)
% 2. Yêu cầu phi chức năng: Tần số hoạt động, tài nguyên FPGA (LUT, BRAM) tối đa cho phép, độ chính xác model AI.
% File: 4_phan_tich_va_kien_truc/4.1_phan_tich_yeu_cau.tex
\subsection{Yêu cầu chức năng}
Để đảm bảo mục tiêu xây dựng một hệ thống SoC hoàn chỉnh có khả năng xử lý trí tuệ nhân tạo tại biên, thiết kế cần đáp ứng bốn nhóm yêu cầu chức năng cốt lõi liên quan đến thu thập dữ liệu, tính toán chuyên dụng, giao tiếp hệ thống và hiệu năng vận hành.

Thứ nhất, đối với phân hệ xử lý hình ảnh, hệ thống được yêu cầu phải có khả năng thu thập dữ liệu video liên tục từ Camera thông qua giao diện song song \textbf{DVP} (Digital Video Port). Luồng dữ liệu này cần được đồng bộ hóa và chuyển đổi định dạng màu sắc để hiển thị trực tiếp lên màn hình qua chuẩn \textbf{HDMI} với độ phân giải tối thiểu là VGA (640x480) hoặc HD (1280x720). Yêu cầu quan trọng đặt ra là quá trình hiển thị phải diễn ra song song với quá trình xử lý, đảm bảo người dùng có thể quan sát hình ảnh thời gian thực với tốc độ khung hình ổn định từ 30 đến 60 fps.

Thứ hai, về năng lực tính toán, hệ thống phải tích hợp một bộ gia tốc phần cứng \textbf{CNN Accelerator} đóng vai trò là một thiết bị ngoại vi chuyên dụng (Memory-mapped Peripheral). Khối này chịu trách nhiệm thực thi các phép toán nhân chập (Convolution) và các hàm kích hoạt phi tuyến của mạng nơ-ron sâu. Accelerator cần có cơ chế truy cập trực tiếp vào bộ nhớ chứa dữ liệu ảnh đầu vào mà không làm gián đoạn luồng video đang hiển thị, đồng thời trả về kết quả phân lớp để vi xử lý tổng hợp.

Thứ ba, để đảm bảo tính tương thích và khả năng mở rộng như một vi điều khiển thương mại, SoC cần hỗ trợ đầy đủ các giao thức giao tiếp tiêu chuẩn công nghiệp. Cụ thể, giao thức \textbf{UART} được sử dụng cho giao diện dòng lệnh (CLI) và gỡ lỗi hệ thống; giao thức \textbf{I2C} đóng vai trò kênh điều khiển cấu hình cho các chip ngoại vi như Camera và HDMI PHY; và giao thức \textbf{SPI/OSPI} được tích hợp để giao tiếp với bộ nhớ Flash hoặc bộ nhớ RAM mở rộng (tốc độ từ 25MHz đến 100MHz), phục vụ cho việc lưu trữ trọng số mạng và chương trình cơ sở (Firmware).

Thứ tư, về chiến lược quản lý xung nhịp và hiệu năng, hệ thống được yêu cầu thiết kế theo kiến trúc đa miền tần số (Multi-Clock Domains) nhằm tối ưu hóa tài nguyên cho từng phân hệ cụ thể. Miền xung nhịp trung tâm (System Clock) điều khiển vi xử lý RISC-V và bộ gia tốc CNN được đặt mục tiêu hoạt động ở tần số \textbf{200 MHz}, đảm bảo thông lượng tính toán cao nhất cho các tác vụ AI. Đối với phân hệ Video Streaming, kiến trúc xung nhịp được phân chia thành ba tầng xử lý riêng biệt: mức \textbf{150 MHz} dành cho các khối xử lý dữ liệu video băng thông rộng và giao tiếp bộ nhớ; mức \textbf{50 MHz} và \textbf{25 MHz} phục vụ cho các giao diện hiển thị và đồng bộ hóa tín hiệu Pixel Clock theo chuẩn VESA. Việc giao tiếp giữa miền 200 MHz của SoC và các miền tần số video thấp hơn phải được thực hiện thông qua các bộ đệm FIFO bất đồng bộ và cơ chế đồng bộ hóa Clock Domain Crossing(CDC) để triệt tiêu hiện tượng Metastability.  

\subsection{Yêu cầu phi chức năng}
Bên cạnh các chức năng vận hành cơ bản, hệ thống phải tuân thủ các ràng buộc kỹ thuật nghiêm ngặt về hiệu năng thời gian thực, tần số hoạt động và quản lý tài nguyên trên nền tảng FPGA đích.

Thứ nhất, về hiệu năng xử lý, hệ thống phải đảm bảo tốc độ khung hình hiển thị ổn định ở mức \textbf{60 FPS} (khung hình/giây) tại độ phân giải mục tiêu. Độ trễ suy luận (Inference Latency) của mô hình AI phải được tối thiểu hóa để kết quả nhận dạng (như nhãn, khung bao) xuất hiện đồng bộ với vật thể đang chuyển động trên màn hình, triệt tiêu hiện tượng trễ pha (Lag) giữa hình ảnh thực tế và kết quả xử lý.

Thứ hai, về tần số hoạt động, thiết kế phải thỏa mãn các chỉ tiêu khắt khe của kiến trúc đa miền xung nhịp. Cụ thể, sau quá trình tổng hợp và hiện thực (Implementation), miền xung nhịp trung tâm (System Clock) cho vi xử lý và bộ gia tốc phải đạt tần số hoạt động ổn định \textbf{200 MHz} để tối đa hóa thông lượng tính toán. Các miền xung nhịp phụ trợ cho video (150 MHz, 50 MHz, 25 MHz) phải đảm bảo sự chính xác về định thời (Timing constraints) để duy trì sự ổn định của tín hiệu hiển thị và giao tiếp bộ nhớ.

Thứ ba, về mặt tài nguyên, thiết kế được tối ưu hóa cho nền tảng bo mạch \textbf{Xilinx VC707} (sử dụng chip Virtex-7 XC7VX485T). Mặc dù đây là dòng FPGA hiệu năng cao với tài nguyên logic dồi dào, thách thức lớn nhất nằm ở việc quản lý hiệu quả băng thông bộ nhớ. Hệ thống phải điều phối chặt chẽ quyền truy cập giữa ba tác nhân tiêu thụ băng thông lớn: Vi xử lý (nạp lệnh/dữ liệu), Bộ gia tốc (đọc/ghi ma trận đặc trưng) và Bộ điều khiển hiển thị (quét bộ đệm khung hình). Việc tối ưu hóa này nhằm đảm bảo luồng video 60 FPS luôn mượt mà ngay cả khi bộ gia tốc hoạt động ở mức tải cao nhất.

\section{Kiến trúc tổng thể SoC}
% Phần này quan trọng nhất:
% - Vẽ sơ đồ khối Top-level (Block Diagram).
% - Mô tả kiến trúc Bus (AXI Interconnect): Ai là Master (CPU, DMA)? Ai là Slave (RAM, Accelerator, Peripherals)?
% File: 4_phan_tich_va_kien_truc/4.2_kien_truc_tong_the.tex

\subsection{Tổng quan kiến trúc SoC}
Để hiện thực hóa các yêu cầu phân tích nêu trên, đề tài đề xuất kiến trúc hệ thống \textbf{SoC không đồng nhất (Heterogeneous SoC)}, kết hợp giữa tính linh hoạt trong điều khiển của vi xử lý mềm (Soft-core Processor) và sức mạnh tính toán song song của phần cứng chuyên dụng.

\begin{figure}[H]
    \centering
    % Thay hình sơ đồ khối SoC của bạn vào đây
    \includegraphics[width=1\linewidth]{3_phan_tich_va_kien_truc/image/AISoC-SoC.drawio.png} 
    \caption{Sơ đồ mô-đun kiến trúc tổng thể của hệ thống SoC RISC-V EdgeAI}
    \label{fig:soc_block_diagram}
\end{figure}

Hệ thống được tổ chức thành ba phân hệ chính hoạt động phối hợp chặt chẽ. Đầu tiên là \textbf{Center Processing Unit} với trung tâm là lõi vi xử lý PicoRV32. Phân hệ này đóng vai trò bộ não của hệ thống, chịu trách nhiệm khởi tạo, cấu hình các ngoại vi và quản lý giao tiếp người dùng.

Tiếp theo là \textbf{Image Detector}, bao gồm khối Accelerator được thiết kế tùy biến để thực thi các phép toán nhân chập (Convolution) nặng nề nhất trong mạng nơ-ron và khối Video Streaming để quản lý luồng video vào từ Camera và hiện thị ra HDMI. 

Cuối cùng là \textbf{Peripherals}, tập hợp các mô-đun giao tiếp và lưu trữ thiết yếu để đảm bảo tính hoàn chỉnh của một hệ thống máy tính nhúng. Phân hệ này tích hợp các bộ điều khiển giao diện chuẩn công nghiệp như \textbf{UART} cho mục đích gỡ lỗi và \textbf{I2C} để cấu hình tham số phần cứng. Đối với giao tiếp lưu trữ, hệ thống áp dụng kiến trúc phân tầng. Trước hết, bộ điều khiển \textbf{SPI} được sử dụng để kết nối với các thiết bị lưu trữ thứ cấp phổ biến như thẻ nhớ SD Card, phục vụ việc lưu trữ dữ liệu ảnh mẫu, chương trình điều khiển(Firmware) hoặc logs hệ thống. Tuy nhiên, để đáp ứng nhu cầu truy xuất băng thông lớn cho trọng số mạng nơ-ron, thiết kế tích hợp thêm mô-đun giao tiếp bộ nhớ tốc độ cao \textbf{OSPI}. Mô-đun này hỗ trợ các chế độ truyền dẫn tiên tiến (Octal-SPI hỗ trợ \textbf{DDR}), giúp tăng tốc độ truy suất bộ nhớ bên ngoài hiệu quả, giải quyết bài toán giới hạn phải xài tài nguyên bộ nhớ nội bộ (BRAM) trên FPGA.

Để đáp ứng yêu cầu khắt khe về định thời trong các ứng dụng thời gian thực, mô-đun \textbf{Timer} được thiết kế với độ chính xác cao dựa trên xung nhịp hệ thống. Chức năng của mô-đun là tạo ra các khoảng trễ (Delay) chính xác cho các giao thức giao tiếp hoặc dùng đề làm Software Timer.

Cuối cùng, mô-đun \textbf{GPIO} cung cấp giao diện điều khiển linh hoạt ở cấp độ bit. Mặc dù có cấu trúc đơn giản, GPIO đóng vai trò không thể thiếu trong việc tương tác trực tiếp với người dùng thông qua hệ thống đèn LED báo trạng thái và nút nhấn điều khiển. Ngoài ra, các chân GPIO còn được quy hoạch để điều khiển các tín hiệu phần cứng quan trọng như tín hiệu Reset cứng cho Camera hay tín hiệu kích hoạt cho màn hình, đảm bảo quy trình khởi động và vận hành của các phân hệ diễn ra theo đúng trình tự thiết kế.



\subsection{Tổ chức hệ thống Bus phân tầng}
Thách thức lớn nhất trong thiết kế này là giải quyết sự tranh chấp băng thông bộ nhớ giữa vi xử lý, bộ gia tốc AI và luồng video thời gian thực. Để khắc phục vấn đề này, kiến trúc Bus được thiết kế theo mô hình \textbf{Bus phân tầng (Hierarchical Bus Architecture)}.

Tầng thứ nhất là \textbf{Bus Ngoại vi (Peripheral Bus)}, sử dụng giao thức \textbf{AXI4-Lite} (biểu diễn bằng các đường kết nối màu xanh dương trong sơ đồ) để kết nối CPU với các ngoại vi bao gồm UART, I2C Master, SPI/OSPI Controller, Timer và GPIO. Các giao dịch trên tuyến bus này chủ yếu là các lệnh đọc/ghi thanh ghi cấu hình, do đó không yêu cầu băng thông lớn và vi xử lý đóng vai trò là Master (AXI4-Lite có hỗ trợ Multi Master và Multi Slave).

Tầng thứ hai là \textbf{Bus Dữ liệu Tốc độ cao (High-Performance Bus)}, được hiện thực chủ yếu dựa trên giao thức \textbf{AXI4-Stream} (biểu diễn bằng các đường kết nối màu xám trong sơ đồ). Đây là giao thức truyền dẫn dòng dữ liệu một chiều không cần địa chỉ, cho phép loại bỏ các chu kỳ trễ (Latency) phát sinh do quá trình bắt tay địa chỉ, từ đó tối đa hóa băng thông cho hệ thống. Tuyến bus này kết nối trực tiếp các thành phần tiêu thụ dữ liệu lớn thông qua cơ chế truy cập bộ nhớ trực tiếp (DMA): Bộ điều khiển Camera (Video DMA Write), Bộ điều khiển truy suất bộ nhớ (DRAM DMA) và Bộ gia tốc AI.

% \subsection{Cơ chế quản lý luồng dữ liệu chia sẻ}
% Để giải quyết bài toán truy cập dữ liệu đồng thời cho hai tác vụ có đặc thù khác nhau là hiển thị (yêu cầu thời gian thực) và tính toán AI (yêu cầu thông lượng cao), hệ thống áp dụng cơ chế \textbf{Bộ đệm khung hình trung tâm (Shared Frame Buffer)}.

% Trung tâm của cơ chế này là một vùng nhớ đệm (Frame Buffer) nằm trong bộ nhớ Bram, đóng vai trò là điểm tập kết dữ liệu duy nhất. Quá trình luân chuyển dữ liệu được thực hiện qua hai tuyến độc lập:

% \textbf{Tuyến Video (Video Path):} Được quản lý bởi \textbf{Bộ điều khiển Video (Video Controller)}. Khối này đảm nhiệm vai trò kép:
% \begin{itemize}
%     \item \textit{Luồng Ghi (Write Channel):} Tiếp nhận dữ liệu thô từ Camera, thực hiện đóng gói và ghi trực tiếp vào Frame Buffer.
%     \item \textit{Luồng Đọc hiển thị (Read Channel):} Tự động đọc dữ liệu từ Frame Buffer theo quy tắc quét dòng (Raster Scan) và gửi đến cổng HDMI. Quá trình này được thực hiện liên tục và đồng bộ với xung nhịp Pixel Clock để đảm bảo tín hiệu hình ảnh không bị gián đoạn.
% \end{itemize}

% \textbf{Tuyến Tăng tốc (Acceleration Path):} Được quản lý bởi một \textbf{DMA riêng biệt (Dedicated DMA)}.
% Khác với luồng hiển thị phải tuân theo định thời VESA, bộ gia tốc (Accelerator) cần dữ liệu đầu vào dưới dạng các khối (Blocks) hoặc dòng (Lines) với tốc độ nhanh nhất có thể. Do đó, một kênh DMA riêng được thiết lập để đọc dữ liệu từ chính Frame Buffer nói trên và chuyển trực tiếp vào bộ nhớ cục bộ của Accelerator. Việc sử dụng DMA riêng giúp tách biệt hoàn toàn miền thời gian của việc hiển thị và việc tính toán, cho phép bộ gia tốc hoạt động ở hiệu suất tối đa mà không ảnh hưởng đến chất lượng hình ảnh đầu ra.


\section{Đặc tả các khối chức năng chính}
% Tại đây chốt thông số kỹ thuật (Spec) cho các khối để Chương 4 và 5 cứ thế mà làm:
% - CPU: Dùng core nào? (VD: PicoRV32, Ibex), tập lệnh gì (RV32I hay RV32IMC)?
% - Accelerator: Kích thước mảng tính toán (VD: 3x3 PE Array), hỗ trợ lớp nào (Conv, Pool, ReLU)?
% - DMA: Số kênh truyền (Channels)?
% - Ngoại vi: Camera độ phân giải bao nhiêu? UART baudrate bao nhiêu?
 % File: 4_phan_tich_va_kien_truc/4.3_dac_ta_khoi.tex

\subsection{Vi xử lý trung tâm (Central Processing Unit)}
Đóng vai trò là bộ não điều phối toàn bộ hoạt động của hệ thống SoC, phân hệ này được xây dựng xung quanh lõi vi xử lý mềm \textbf{PicoRV32} - một hiện thực tối ưu về tài nguyên của kiến trúc tập lệnh RISC-V chuẩn RV32I. Lõi vi xử lý này chịu trách nhiệm thực thi các tác vụ điều khiển logic chính, từ việc khởi tạo hệ thống đến quản lý các ngoại vi.

Để hỗ trợ hoạt động của CPU, kiến trúc bộ nhớ cục bộ được tổ chức thành ba vùng riêng biệt nhằm tối ưu hóa hiệu năng truy xuất:
\begin{itemize}
    \item[] \textbf{Bộ nhớ Khởi động (BMEM - Bootloader Memory):} Đây là thành phần quan trọng chứa các tập lệnh khởi động cơ bản (Boot ROM). Ngay khi hệ thống được cấp nguồn hoặc reset, CPU sẽ trỏ thanh ghi bộ đếm chương trình (PC) vào vùng nhớ này đầu tiên. Nhiệm vụ của Bootloader là thiết lập các thông số phần cứng ban đầu và nạp chương trình chính từ bộ nhớ ngoài (Flash SPI) vào IMEM trước khi trao quyền điều khiển lại cho ứng dụng.
    \item[] \textbf{Bộ nhớ Lệnh (IMEM - Instruction Memory):} Là vùng nhớ chứa mã chương trình chính (Main Application) mà CPU sẽ thực thi sau quá trình khởi động. Vùng nhớ này thường được ánh xạ vào Block RAM để đảm bảo tốc độ truy xuất lệnh nhanh nhất (một chu kỳ máy).
    \item[] \textbf{Bộ nhớ Dữ liệu (DMEM - Data Memory):} Dùng để lưu trữ các biến toàn cục, ngăn xếp (Stack) và dữ liệu tạm thời trong quá trình tính toán của chương trình. Việc tách biệt DMEM và IMEM (kiến trúc Harvard sửa đổi) giúp tránh xung đột khi CPU thực hiện nạp lệnh và truy xuất dữ liệu đồng thời.
    
\end{itemize}

\subsection{Nhận diện Hình ảnh (Image Detector)}
Đây là phân hệ cốt lõi tạo nên tính năng thông minh của hệ thống, chịu trách nhiệm thực hiện song song hai tác vụ: duy trì luồng hình ảnh thời gian thực và thực thi các thuật toán trí tuệ nhân tạo. Cấu trúc của phân hệ này là sự tích hợp chặt chẽ giữa chuỗi xử lý video (Video Streaming Pipeline) và khối tính toán chuyên dụng.

\subsubsection{Hệ thống Video Streaming}
Khối này quản lý dòng chảy dữ liệu hình ảnh liên tục để phục vụ nhu cầu quan sát. Tại ngõ vào, giao diện thu thập dữ liệu tiếp nhận tín hiệu từ Camera qua chuẩn song song DVP, thực hiện đồng bộ và đóng gói dữ liệu vào bộ đệm khung hình (Frame Buffer). Tại ngõ ra, bộ điều khiển hiển thị đọc dữ liệu từ bộ đệm này và chuyển đổi thành tín hiệu chuẩn HDMI, đảm bảo xuất hình ảnh mượt mà lên màn hình với độ trễ tối thiểu.

\subsubsection{Khối Gia tốc (Accelerator)}
Đóng vai trò là trung tâm xử lý AI chuyên trách xử lý các phép toán nhân chập (Convolution) nặng nề của mạng nơ-ron.

Để đảm bảo khả năng cung cấp dữ liệu liên tục cho mảng tính toán mà không làm nghẽn bus hệ thống, khối gia tốc được tích hợp cơ chế truy xuất bộ nhớ băng thông rộng thông qua hai kênh DMA chuyên biệt:
\begin{itemize}
    \item[] \textbf{DMA 0 (Data/Weight Reader):} Kênh này chịu trách nhiệm đọc các bộ trọng số (Weights) từ bộ nhớ hệ thống (Frame Buffer hoặc Weight Memory) để nạp vào bộ đệm nội của Accelerator.
    \item[] \textbf{DMA 1 (Result Writer):} Kênh này chịu trách nhiệm đọc dữ liệu đặc trưng đầu vào (Input Feature Maps) và thu thập kết quả tính toán (Output Feature Maps) từ khối Image Detector và ghi ngược trở lại bộ nhớ chính, sẵn sàng cho các lớp xử lý tiếp theo hoặc để vi xử lý trung tâm đọc kết quả phân lớp.
\end{itemize}


\subsection{Các Ngoại vi (Peripherals)}
Phân hệ Ngoại vi tích hợp các khối chức năng chuẩn hóa, cung cấp các giao thức giao tiếp phổ biến để đảm bảo khả năng tương thích và mở rộng cho hệ thống nhúng:

\begin{itemize}
    \item[] \textbf{UART:} Cung cấp giao thức truyền thông nối tiếp không đồng bộ (Asynchronous Serial Communication), phục vụ việc trao đổi dữ liệu dòng và hỗ trợ giao diện gỡ lỗi hệ thống.
    \item[] \textbf{I2C:} Cung cấp giao thức giao tiếp nối tiếp hai dây (Two-wire Interface), đóng vai trò Master điều khiển và cấu hình các thiết bị ngoại vi tham gia vào bus hệ thống.
    \item[] \textbf{SPI/OSPI:} Cung cấp giao thức truyền thông nối tiếp đồng bộ tốc độ cao (Serial Peripheral Interface), hỗ trợ mở rộng kết nối với các bộ nhớ ngoài hoặc các thiết bị ngoại vi yêu cầu băng thông truyền tải lớn.
    \item[] \textbf{Timer \& GPIO:} Cung cấp tài nguyên định thời gian thực cho hệ thống và các giao diện điều khiển tín hiệu số vào/ra đa mục đích (General Purpose Input/Output).
\end{itemize}

\section{Tổ chức bộ nhớ và Bản đồ địa chỉ (Memory Map)}
% Bảng quy hoạch địa chỉ (Address Map) là bắt buộc trong thiết kế SoC.
% VD: RAM từ 0x0000_0000, Accelerator từ 0x4000_0000, UART từ 0x8000_0000...
% Phần này giúp Software (Firmware) biết địa chỉ nào để điều khiển phần cứng.
% File: 4_phan_tich_va_kien_truc/4.4_memory_map.tex

\subsection{Khái niệm và vai trò của Memory Map}
\textbf{Bản đồ bộ nhớ (Memory Map)} là một cấu trúc dữ liệu mô hình hóa cách thức hệ thống phân bổ các địa chỉ số (thường dưới dạng hệ thập lục phân - Hexadecimal) vào các tài nguyên phần cứng vật lý trong hệ thống SoC. Trong kiến trúc xử lý, vi xử lý PicoRV32 không tương tác trực tiếp với các thiết bị ngoại vi bằng tên gọi, mà thông qua một không gian địa chỉ phẳng duy nhất.



Việc quy hoạch bản đồ bộ nhớ là bước thiết kế tiên quyết vì những lý do sau:
\begin{itemize}
    \item[] \textbf{Thống nhất giao tiếp (Memory-mapped I/O):} Cho phép CPU coi các thanh ghi điều khiển của ngoại vi (như UART, I2C) tương tự như các ô nhớ thông thường. Điều này giúp đơn giản hóa tập lệnh của vi xử lý vì chỉ cần các lệnh nạp/lưu dữ liệu ($Load/Store$) để điều khiển toàn bộ phần cứng.
    \item[] \textbf{Định tuyến dữ liệu (Address Decoding):} Cung cấp thông tin cho bộ giải mã địa chỉ (Address Decoder) trong khối \textbf{AXI Interconnect}. Dựa trên địa chỉ mà CPU phát ra, hệ thống sẽ biết chính xác cần kích hoạt tín hiệu chọn thiết bị ($Chip Select$) nào để dẫn luồng dữ liệu đến đúng đích.
    \item[] \textbf{Tránh xung đột tài nguyên:} Đảm bảo mỗi thành phần phần cứng được cấp phát một vùng không gian riêng biệt, không chồng lấn, từ đó triệt tiêu các lỗi xung đột địa chỉ khi hệ thống vận hành.
    \item[] \textbf{Cơ sở cho phát triển phần mềm (Firmware):} Bản đồ bộ nhớ cung cấp các địa chỉ cơ sở ($Base Address$) giúp người lập trình xây dựng các trình điều khiển thiết bị (Drivers) và cấu hình trình biên dịch (Linker Script) để nạp mã nguồn vào đúng vị trí trong bộ nhớ.
\end{itemize}

Dựa trên kiến trúc SoC đề xuất, không gian địa chỉ được chia thành hai phân vùng lớn: Vùng nhớ hệ thống (System Memory) và Vùng địa chỉ ngoại vi (Peripherals).

\subsection{Bản đồ vùng nhớ hệ thống}
Vùng nhớ hệ thống bao gồm các khối BRAM chứa mã thực thi và dữ liệu hoạt động của vi xử lý. Chi tiết phân bổ được trình bày trong Bảng \ref{tab:system_memory_map}.

\begin{table}[H]
    \centering
    \caption{Bản đồ địa chỉ vùng nhớ hệ thống (System Memory Map)}
    \label{tab:system_memory_map}
    \renewcommand{\arraystretch}{1.4}
    \begin{tabular}{|p{2.5cm}|p{4.5cm}|p{7.5cm}|}
        \hline
        \textbf{Thành phần} & \textbf{Dải địa chỉ (Hex)} & \textbf{Mô tả Chức năng} \\ 
        \hline
        \textbf{DMEM} & \texttt{0x0000\_0000}  \texttt{0x0001\_0000} & \textbf{Data Memory (64KB)}. Vùng nhớ dữ liệu, Stack, Heap \\ 
        \hline
        \textbf{BMEM} & \texttt{0x0100\_0000}  \texttt{0x0101\_0000} & \textbf{Boot Memory (64KB)}. Chứa mã khởi động (Bootloader). \\ 
        \hline
        \textbf{IMEM} & \texttt{0x0110\_0000}  \texttt{0x0111\_0000} & \textbf{Instruction Memory (64KB)}. Vùng nhớ chứa mã lệnh chương trình chính (Firmware). \\ 
        \hline
        
    \end{tabular}
\end{table}

\subsection{Bản đồ vùng ngoại vi}
Vùng ngoại vi bắt đầu từ địa chỉ cơ sở \texttt{0x8000\_0000}. Mỗi ngoại vi được cấp phát một không gian 4KB (Offset \texttt{0x1000}) để chứa các thanh ghi cấu hình. Chi tiết được trình bày trong Bảng \ref{tab:peripheral_map}.

\begin{table}[H]
    \centering
    \caption{Bản đồ địa chỉ vùng ngoại vi (Peripheral Memory Map)}
    \label{tab:peripheral_map}
    \renewcommand{\arraystretch}{1.4}
    \begin{tabular}{|p{2.5cm}|p{4.5cm}|p{7.5cm}|}
        \hline
        \textbf{Thành phần} & \textbf{Dải địa chỉ (Hex)} & \textbf{Mô tả Chức năng} \\ 
        \hline
        \textbf{GPIO} & \texttt{0x8000\_0000}  \texttt{0x8000\_0FFF} & Điều khiển các tín hiệu vào/ra cơ bản (LEDs, Buttons). \\ 
        \hline
        \textbf{UART} & \texttt{0x8000\_1000}  \texttt{0x8000\_1FFF} & Bộ điều khiển giao tiếp nối tiếp (Console/Debug). \\ 
        \hline
        \textbf{I2C} & \texttt{0x8000\_2000}  \texttt{0x8000\_2FFF} & Giao tiếp cấu hình Camera và chip HDMI PHY. \\ 
        \hline
        \textbf{SPI} & \texttt{0x8000\_3000}  \texttt{0x8000\_3FFF} & Giao tiếp thẻ nhớ SD Card hoặc Flash phụ trợ. \\ 
        \hline
        \textbf{OSPI} & \texttt{0x8000\_4000}  \texttt{0x8000\_4FFF} & Giao tiếp bộ nhớ tốc độ cao (Octal-SPI/DDR). \\ 
        \hline
        \textbf{Timer} & \texttt{0x8000\_5000}  \texttt{0x8000\_5FFF} & Bộ định thời gian thực và đo đạc hiệu năng. \\ 
        \hline

    \end{tabular}
\end{table}

Cơ chế giải mã địa chỉ được thực hiện bởi bộ \textbf{AXI Interconnect}, đảm bảo tín hiệu chọn thiết bị tớ (Slave Select) được gửi chính xác đến từng khối chức năng dựa trên địa chỉ mà CPU phát ra trên bus hệ thống.